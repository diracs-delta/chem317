\documentclass[journal = jacsat, manuscript = article, layout = twocolumn]{achemso}

\usepackage[version = 4]{mhchem}
\usepackage{graphicx}
\usepackage{mwe}
\usepackage{gensymb}
\usepackage[labelfont=bf]{caption}
\usepackage{booktabs}

% disable symbol next to email
\makeatletter
\def\acs@author@fnsymbol#1{}
\makeatother
% enable abstract
\AbstractOn
% put abstract and title on the same page
\let\oldmaketitle\maketitle
\let\maketitle\relax
% shut up latex
\hbadness=99999

% disable symbol next to email
\makeatletter
\def\acs@author@fnsymbol#1{}
\makeatother

% macros
\newcommand{\h}{$^1$H}
\newcommand{\p}{$^{31}$P}
\newcommand{\cdcl}{\ce{CDCl3}}
\newcommand{\del}{$\delta = $}
\newcommand{\wavenum}{cm$^{-1}$}
\newcommand{\kmn}{\ce{KMnO4}}
\newcommand{\chcl}{\ce{CHCl3}}
\newcommand{\acac}[1]{\ce{#1(acac)3}}
\newcommand{\acactwo}[1]{\ce{#1(acac)2}}

\title{Identification of Unknown Metal Acetylacetonate Complexes via Evan's Method}

\author{David Qiu}
\affiliation{Department of Chemistry, University of Illinois at
Urbana-Champaign, 505 S Matthews Avenue, Urbana, IL, 61801}
\email{davidlq2@illinois.edu}

\begin{document}

\twocolumn[
\begin{@twocolumnfalse}
\oldmaketitle
\hrule
\begin{abstract}

	Evan's method is a promising analytical technique for detection of
	paramagnetic metal complexes, with applications for pollutant detection
	in environmental chemistry. This experiment demonstrated successful
	synthesis of \acac{Mn} (41\% yield), calculation of magnetic moments via
	\h\ NMR and Evan's method, and qualitative determination of unknown
	compounds. The unknown compounds (Unknowns 1--4) were identified as
	follows: low-spin \acac{Co}, \acac{Cr}, high-spin \acac{Fe}, and
	\acactwo{Cu}, with corresponding magnetic moments (in units of
	$\mu_\text{B}$) 0, 2.56, 2.99, and 1.96 respectively. Future studies
	should improve upon this work by using more vigorous stirring and finer
	filters during synthesis and filtration of \acac{Mn}, by accounting for
	diamagnetic corrections and LS-coupling, and by properly shimming and
	calibrating NMR instruments prior to analysis via Evan's method.

\end{abstract}
\hrule
\vspace{6mm}%
\end{@twocolumnfalse}
]

\section{Introduction}

Paramagnetism arises in inorganic complexes that possess unpaired electrons,
which allow it to generate a magnetic field parallel to an external magnetic
field. This property causes metal complexes to generate unique NMR spectra due
to their influence on the internal magnetic field, and can be exploited to
determine the identity of any paramagnetic complex. One such experimental
technique is \emph{Evan's method}, which utilizes a sealed capillary of
``control" diamagnetic solution, attaining an NMR spectrum of the (paramagnetic)
metal complex solution with the capillary submerged, and determining the
identity of the complex via the difference in the chemical shifts $\Delta f$ of
\chcl\ in different magnetic environments.

Upon determination of $\Delta f$ (in Hz), the \emph{magnetic susceptibility}
$\chi_M$ is related by the expression \cite{handout}

\begin{equation}
	\chi_M = \frac{3 \Delta f}{4 \pi F c},
\end{equation}

where $F$ is the field strength (in Hz) and $c$ is the molar concentration of
the complex, in mol/mL. $\chi_M$ is in turn related to the \emph{magnetic
moment} $\mu$ by the relation

\begin{equation}
	\mu = \sqrt{8 \chi_M T},
\end{equation}

where $T$ is the temperature in K.

Evan's method is thus especially useful for determination of both the presence
and identity of inorganic metal complexes. This could be exploited in future
studies to yield a novel method of detecting transition metal pollutants in the
environment, which have become increasingly prevalent throughout the ecosystem,
and have demonstrated a serious cause for concern in both our ecology and public
health, even at sublethal concentrations. \cite{pollution1, pollution2} While
analytical techniques for pollutant detection certainly exist and have advanced
considerably in the past decade, \cite{detection1, detection2, detection3}  they
often demonstrate a lack of large-scale extensibility and a burdensome economic
cost.  Evan's method thus presents as a promising direction for pollutant
detection.

\begin{figure}
	\ce{MnO4^- + H^+ + 3 Hacac $\hspace{3cm}$ \\
	$\hspace{2cm}$ ->[100 \degree C][5 min] Mn(acac)3 + 2 H2O + O2}

	\caption*{\textbf{Scheme 1:} Synthesis of \acac{Mn} via reduction of
	\kmn\ in a boiling aqueous solution of acetylacetone.}
\end{figure}

In this experiment, \acac{Mn} was synthesized via aqueous reduction of \kmn\ in
the presence of acetylacetone (Scheme 1), and Evan's method was used to
determine the magnetic moments of \acac{Mn} and four other unknown metal
acetylacetonate complexes, hereafter labelled Unknowns 1--4. The magnetic
moments, determined via \h\ NMR, were then used to identify the unknowns and
assign them to a transition metal complex. Metal acetylacetonate complexes are
an important reagent in many industrial reactions \cite{acac_uses}, and were
thus chosen to demonstrate the efficacy of Evan's method in their detection and
identification. Furthermore, acetylacetonate complexes have a higher likelihood
of being paramagnetic, as their low ligand field strength favors formation of
high-spin complexes.\cite{textbook_2}

\section{Results and Discussion}

A percent yield of 41\% was recorded for the synthesis of \acac{Mn}. This yield
is quite low, and could have resulted from either failure to vigorously stir the
reaction mixture or from loss of product during the coarse glass frit filtration
step. It is unlikely that there is a thermodynamic or kinetic effect limiting
the yield, as literature studies reporting the exact same reaction times,
temperatures, and molar ratios report a percent yield of 87\%.
\cite{mnacac_yield} Future studies wishing to improve yields should use more
vigorous stirring and use a more fine filtration apparatus.

The unknown samples were all observed to be strongly colored first-row
transition metals. Furthermore, all unknowns were tris-chelated except Unknown
4, which was bis-chelated. All unknowns except Unknown 1 exhibited a
paramagnetic shift in their respective \h\ NMR spectra. A summary of
observations is tabulated in Table 1.

\begin{table}[]
\begin{tabular}{@{}lll@{}}
\toprule
Sample         & Color     & $\Delta f$ (Hz) \\ \midrule
\ce{Mn(acac)3} & Purple    & 1320            \\
Unk.\ 1        & Green     & 0               \\
Unk.\ 2        & Violet    & 635             \\
Unk.\ 3        & Crimson   & 1700            \\
Unk.\ 4        & Blue-Gray & 345             \\ \bottomrule
\end{tabular}
\caption{Summary of observations for each sample.}
\end{table}

A thorough determination of the unknown complexes is non-trivial, as the molar
concentration of the paramagnetic solution is unknown as the molar mass is
unknown, and thus Equations 1 and 2 cannot be applied directly. $\mu$ must be
computed for each element of interest, and then compared with the expected $\mu$
of the corresponding complex. For instance, one would first assume an unknown to
be \acac{M}, and compute $\mu$ using Equations 1 and 2, compare the calculated
result with the expected value for both the corresponding low and high-spin
complexes, and repeat this for all possible complexes. The complex which
minimizes the absolute difference $|\Delta\mu| = |\mu - \mu_\text{exp}|$ is the
identity of the unknown as evidenced by Evan's method. The summary of these
results and their assignments are tabulated below.

\tabcolsep = 0.11cm
\begin{table}[]
\begin{tabular}{@{}llll@{}}
\toprule
Sample         & $\mu$  & $|\Delta\mu|$  & Assignment            \\ \midrule
\ce{Mn(acac)3} & 4.77                   & 0.12                         & --                  \\
Unk.\ 1        & 0                      & 0                            & Low-spin \acac{Co}  \\
Unk.\ 2        & 2.56                   & 0.85                         & \acac{Cr}           \\
Unk.\ 3        & 2.99                   & 2.92                         & High-spin \acac{Fe} \\
Unk.\ 4        & 1.96                   & 0.23                         & \acactwo{Cu}        \\ \bottomrule
\end{tabular}
\caption{Tabulated magnetic moments $\mu$ and their absolute difference $|\Delta\mu|$
with respect to their corresponding assignment. All magnetic moments are in
units of $\mu_\text{B}$. $\mu$ was calculated using the molar mass of the
assigned complex.}
\end{table}

Unknown 1 was assigned on the basis of its diamagnetism and color. The only
other diamagnetic tris-chelate is \ce{Sc^{3+}}, which is colorless in solution
due to its lack of d-electrons. \cite{textbook_2} Thus, the only possible
identity of Unknown 1 is the low-spin complex \acac{Co}. This result is
interesting as the acac ligand is expected to generate high-spin complexes due
to its low ligand field strength as mentioned earlier. This shows that in the
extreme case of exactly 6 d-electrons, \acac{Co} preferentially forms a low-spin
complex, which implies that the ligand field stabilization energy (LFSE) is just
slightly lower than that of the exchange energy stabilizing the corresponding
high-spin complex. This observation of low-spin \acac{Co} in the ground state is
in agreement with existing literature on its magnetic properties.
\cite{coacac_diamagnetic}

Unknown 4 was easily assigned to \acactwo{Cu} on the basis of its similar
magnetic moment and its blue color being that of a Cu(II) solution.
\cite{textbook_2}

Unknowns 2 and 3 demonstrated a high $|\Delta\mu|$ as shown in Table 2,
exhibiting the limitations of reliance purely on Evan's method for qualitative
determination. Unknown 2 was assigned to \acac{Cr} despite its calculated
magnetic moment being more similar to that of \acac{Mn} because of its
similarity in color to the \acac{Cr} complex. \cite{cracac_color, feacac_color}
Similarly, Unknown 3 exhibited the famous crimson-red color of a Fe(III)
solution, whose corresponding acetylacetonate complex is known to be high-spin.
\cite{feacac_color}

Unknowns 2 and 3 demonstrate different magnetic moments with respect to their
assignments mainly due to poor \h\ NMR spectra (presented in the Supporting
Information). Repetition of the same data analysis on literature values of the
paramagnetic shifts \cite{handout} and another classmate's data (not included)
demonstrate very similar magnetic moments to those of \acac{Cr} and
high-spin \acac{Fe} respectively. Thus, future studies should attempt to properly
calibrate and shim the \h\ NMR instrument prior to analysis via Evan's method.

Another, less pertinent source of error arises in simplication. Equation 1 fails
to account for the diamagnetic correction necessary for larger complexes,
\cite{handout} and Equation 2 also fails to account for the magnetic moment
arising from LS-coupling observed in the heavier transition metal elements.
\cite{handout, textbook, textbook_2} These two effects, in conjunction, could
have led to the anomalously low or high calculated magnetic moments determined
via \h\ NMR and Evan's method.

\section{Conclusion}

In conclusion, this experiment successively demonstrated synthesis of \acac{Mn},
calculation of magnetic moments via \h\ NMR and Evan's method, and qualitative
determination of unknown compounds. The yield of \acac{Mn} was determined to be
lower than those of literature under the same reaction conditions, and thus
future studies should ensure vigorous stirring and usage of a more fine
filtration apparatus. Direct application of Evan's method was complicated by
both failure to account for diamagnetic corrections and LS-coupling, and by the
poor \h\ NMR data due to improper shimming and calibration prior to analysis.
Future studies should improve on this by using more accurate formulas for the
magnetic moment (c.f.\ Girolami et al.) and ensuring proper calibration by
visualizing correct peak shapes in \h\ NMR spectra prior to analysis via Evan's
method.

\bibliography{lab_5.bib}

\end{document}
