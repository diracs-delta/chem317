\documentclass[journal = jceda8, manuscript = article]{achemso}

\usepackage[version = 4]{mhchem}
\usepackage{graphicx}
\usepackage{mwe}
\usepackage{textcomp, gensymb}
\usepackage[labelfont=bf]{caption}
\usepackage{amsmath}

% disable symbol next to email
\makeatletter
\def\acs@author@fnsymbol#1{}
\makeatother
% shut up latex
\hbadness=99999

% macros
\newcommand{\h}{$^1$H}
\newcommand{\p}{$^{31}$P}
\newcommand{\cdcl}{\ce{CDCl3}}
\newcommand{\wavenum}{cm$^{-1}$}
\newcommand{\del}{$\delta$}
\newcommand{\deleq}[1]{$\delta=#1$}
\newcommand{\hamil}{\hat{\mathcal{H}}}

\title{Synthesis of CdSe Semiconductor Nanocrystals}

\author{David Qiu}
\affiliation{Department of Chemistry, University of Illinois at
Urbana-Champaign, 505 S Matthews Avenue, Urbana, IL, 61801}
\email{davidlq2@illinois.edu}

\begin{document}

\section{Introduction}

% Beginning with an Introduction (15 points) that includes the history and
% mechanism of the reaction. You should explain the importance of the reaction to
% inorganic chemistry and the significance to you - why did you choose this one?
% Furthermore, you should explain the importance of this reaction in the world of
% chemistry in general.

Semiconductor nanostructures translate quantum-mechanical phenomena into unique,
size-dependent, readily tunable electrical and optical
properties.\cite{nano_rev} These hold promising applications for photovoltaics
and green energy, where nanocrystals have already been applied to engineer
highly-efficient, economical solar cells. \cite{solar_1, solar_2, solar_3}
Other inorganic nanostructures have also been found to catalyze \ce{CO2} fixation
\cite{co2_1, co2_2}, serve as fluorescent biological imaging tags, \cite{bio_1,
bio_2} and even form single-electron transistors in nanoelectronics,
\cite{nano_elec} all of which have extensive applications in critical
technologies and industries in our society.

Sadly, the presence of nanochemistry in modern research and industry dwarfs its
presence in the undergraduate curriculum. The wide scope of nanochemistry is
seldom discussed in courses, despite its numerous functions in modern chemistry.
This set of experiments aims to change that by demonstrating modern synthetic
techniques employed to generate nanostructures and educating on the unique
electrical and optical properties rising from the both quantum phenomena of
nanostructures.

CdX (X = S, Se, Te) semiconductor spherical nanocrystals (i.e.\ quantum dots)
display readily apparent optical properties, and are some of the simplest and
most well-studied nanostructures made in the laboratory. Modern one-pot
syntheses involving relatively non-toxic elemental precursors have been
well-established by Peng and colleagues, thus making CdX quantum dots ideal
targets for study in the undergraduate laboratory. \cite{peng_1, peng_2} Both
shape and size are controlled simply by the CdO precursor concentration.  At
moderate CdO concentrations, monodisperse (i.e.\ similar to each other in size)
spherical nanocrystals are replicably and easily synthesized.  This is believed
to occur due to diffusion-dependent crystal growth on certain faces of the
nanocrystal upon heating. \cite{peng_mechanism}

To provide a gentle introduction into the theory behind the optical properties
of nanocrystals, one must first understand what happens during the absorption of
a photon. It is well-established that in semiconductors, absorption of a photon
results in the formation of a electron-hole pair, which can be roughly
visualized as an electron (e) orbiting a positively-charged hole (h).
\cite{semiconductor-text, excitons} This is known as a Wannier-Mott exciton, and
typically has a large corresponding Bohr exciton radius, far greater than that
of the Bohr radii of the constituent atoms.

Thus, the Hamiltonian of a free Wannier-Mott exciton is trivially related by the
expression

\begin{equation}
		\hamil = - \frac{\hbar}{2 m_h} \nabla_h^2
		         - \frac{\hbar}{2 m_e} \nabla_e^2
			 - \frac{e^2}{\epsilon |r_e - r_h|},
\end{equation}

where $m_e$ ($m_h$) represents the effective mass of the electron (hole), and
$\epsilon$ represents the dielectric constant of the semiconductor. This is
analogous to the Hamiltonian of the hydrogen atom. However, this equation must
fail when the system undergoes quantum confinement, i.e.\ the nanocrystal radius
is less than that of the exciton radius, and the exciton is no longer free.
While a full derivation is beyond the scope of this paper, it suffices the
mention that an additional polarization term $\hat V_\text{pol}$ must be
included to account for the polarization of electric charge resulting from the
formation of the exciton, which obeys a strong dependence on the radius of the
spherical nanocrystal $R$. Following the derivation outlined by Brus, the energy
of the lowest excited state of an exciton is approximated by the expression
\cite{exciton_energy}

\begin{equation}
E = E_0 + \frac{\hbar^2\pi^2}{2R^2} \left[ \frac{1}{m_e} + \frac{1}{m_h} \right]
    - \frac{1.8 e^2}{\epsilon R}
    + \frac{e^2}{R} \overline{\sum^{\infty}_{n = 1} \alpha_n \left( \frac{S}{R} \right)^{2n}},
\end{equation}

where $E_0$ is the bulk band gap, $\alpha$ is a constant dependent on the
dielectric constant of the semiconductor and the surrounding solvent, S is the
position of the electron. The bar above the third term denotes that it should be
taken as an integral over the entire wavefunction of the electron.

While this equation is certainly complex, it demonstrates the energy of the
excited state decreasing monotonically with increasing radius. This is exactly
analogous to the particle-in-a-box problem introduced in undergraduate physical
chemistry, which also exhibits an inverse dependence between the energy $E$ and
the size of the box $L$.

\section{Pedagogy}

% Next should be a section on Pedagogy (15 points). What do you want your
% students to learn from this experiment, both in terms of lab techniques and
% broader conceptual aspects.

In this experiment, students will learn the synthesis of CdS

\section{Hazards}

% Then a Hazards (10 points) section - this may be just a paragraph, but clearly
% indicate any chemical or other hazards the students will encounter.

\section{Experimental Procedures}

% Experimental procedure (20 points). This part may be adapted from the literature
% on which you are basing your paper. This should be <50% of the paper in terms of
% length.

\section{Results and Discussion}

% Results and discussion (15 points). This section will discuss a number of
% aspects including: how many lab periods the students will need and how many
% hours for each as well as what will be accomplished in each period. What
% problems are anticipated, what are the average or expected yields, and how will
% the compound(s) be characterized?  Are there any byproducts expected? Each
% experiment should include a characterization section where you detail which
% techniques you will use, multinuclear NMR spectroscopy, IR spectroscopy, Evan’s
% method,

\section{Conclusion}

% Conclusion (10 points). This can be just a paragraph.

\bibliography{bib.bib}

\end{document}
