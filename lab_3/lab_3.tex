\documentclass[journal = jacsat, manuscript = article, layout = twocolumn]{achemso}

\usepackage{mhchem}
\usepackage{graphicx}
\usepackage{mwe}

% disable symbol next to email
\makeatletter
\def\acs@author@fnsymbol#1{}
\makeatother
% enable abstract
\AbstractOn
% put abstract and title on the same page
\let\oldmaketitle\maketitle
\let\maketitle\relax

% macros
\newcommand{\nidppe}{\ce{Ni(dppe)Cl2}}
\newcommand{\nicl}{\ce{NiCl2 $\cdot$ 6H2O}}
\newcommand{\nh}{\ce{NH3}}
\newcommand{\nanh}{\ce{Na $\cdot$ NH3}}
\newcommand{\nhbr}{\ce{NH4Br}}

\title{aaaawubadugh}

\author{David Qiu}
\affiliation{Department of Chemistry, University of Illinois at Urbana-Champaign, 505 S Matthews Avenue, Urbana, IL, 61801}
\email{davidlq2@illinois.edu}

\begin{document}

\twocolumn[
\begin{@twocolumnfalse}
\oldmaketitle
\hrule
\begin{abstract}

\end{abstract}
\hrule
\vspace{6mm}%
\end{@twocolumnfalse}
]

\section{Introduction}

\begin{figure}[H]
	\includegraphics[width=0.5\textwidth]{figures/scheme.png}
	\caption*{Scheme 1: Synthesis of dppe via reduction of triphenylphosphine.}
\end{figure}

Ammonia is a versatile inorganic solvent with the unique property of dissolving
alkali metals to form solvated electrons, which act as extremely strong reducing
agents in solution. In this experiment, \nanh\ was used to reduce a
solution of triphenylphosphine to yield the bidentate ligand
1,2-bis(diphenylphosphino)ethane (dppe). The newly-synthesized dppe ligand was
then used to perform ligand substitution to generate its corresponding nickel
complex. This experiment demonstrates the efficacy of using cheap,
readily-available reagents to synthesize inorganic ligands for use in industry.

\begin{figure}[H]
	\includegraphics[width=0.4\textwidth]{figures/scheme-2.png}
	\caption*{Scheme 2: Synthesis of \nidppe\ via direct ligand substitution of \nicl.}
\end{figure}

\section{Experimental Procedures}

To a 500 mL three-necked round bottom flask was charged 200 mL of \nh, a glass-coated stir bar, and a dry ice condenser. Remaining necks were sealed with stoppers, and 2.379 g of Na was added slowly over the course of 3 minutes. A dark blue solution was allowed to form over the course of 10 minutes, and then 13.55 g of triphenylphosphine was added in small 1 g portions. This solution then reacted for 30 minutes, after which 5.068 g of \nhbr\ was added. Finally, 2.555 g of 1,2-dichloroethane was poured in and was allowed to react for 10 minutes. The flask was left open to air for 1 week to dry.

The dried

\end{document}
